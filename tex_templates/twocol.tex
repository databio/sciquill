

\let\tightlist\relax

\let\Begin\begin
\let\End\end


\providecommand{\tightlist}{%
  \setlength{\itemsep}{0pt}\setlength{\parskip}{0pt}}

  
\documentclass[10pt]{article}
\RequirePackage[top=0.85in,left=0.85in,right=0.85in,bottom=0.85in]{geometry}


\usepackage{authblk}
\renewcommand{\rmdefault}{phv} % Arial
\renewcommand{\sfdefault}{phv} % Arial

\renewcommand{\rmdefault}{bch} % Charter
\renewcommand{\sfdefault}{bch} % Charter


\usepackage{multicol}


\usepackage{pdfpages}
\usepackage{fancyhdr}

\usepackage{wrapfig}
\usepackage[skip=2pt, font=small, format=plain, labelfont=it, textfont=it, tableposition=below]{caption}

% longtable is required for markdown-format tables to be rendered in latex
\usepackage{longtable}
% \usepackage{xtab}




\usepackage{amssymb,amsmath}
\usepackage{ifxetex,ifluatex}

% colors
\definecolor{bluehaze}{RGB}{54,95,145}
\definecolor{maroonhaze}{RGB}{148,54,52}
\definecolor{orangehaze}{RGB}{227,108,10}
\definecolor{black}{RGB}{0,0,0}
\definecolor{bluetint}{RGB}{0,50,100}
\definecolor{graytint}{RGB}{175,175,175}
\definecolor{graybit}{RGB}{245,245,245}


% Turn on shading??
$if(highlighting-macros)$
$highlighting-macros$
$endif$



\ifxetex
  \usepackage{fontspec,xltxtra,xunicode}
  \defaultfontfeatures{Mapping=tex-text,Scale=MatchLowercase}
\else
  \ifluatex
    \usepackage{fontspec}
    \defaultfontfeatures{Mapping=tex-text,Scale=MatchLowercase}
  \else
    \usepackage[utf8]{inputenc}
  \fi
\fi
$if(natbib)$
\usepackage{natbib}
\bibliographystyle{plainnat}
$endif$
$if(biblatex)$
\usepackage{biblatex}
$if(biblio-files)$
\bibliography{$biblio-files$}
$endif$
$endif$
$if(lhs)$
\usepackage{listings}
\lstnewenvironment{code}{\lstset{language=Haskell,basicstyle=\small\ttfamily}}{}
$endif$
$if(verbatim-in-note)$
\usepackage{fancyvrb}
$endif$
$if(fancy-enums)$
% Redefine labelwidth for lists; otherwise, the enumerate package will cause
% markers to extend beyond the left margin.
\makeatletter\AtBeginDocument{%
  \renewcommand{\@listi}
    {\setlength{\labelwidth}{4em}}
}\makeatother
\usepackage{enumerate}

% Attempt to smash lists together
\setlist[enumerate]{itemsep=1mm}

$endif$
$if(tables)$
\usepackage{ctable}
\usepackage{float} % provides the H option for float placement
$endif$
$if(url)$
\usepackage{url}
$endif$
$if(graphics)$
\usepackage{graphicx}
% We will generate all images so they have a width \maxwidth. This means
% that they will get their normal width if they fit onto the page, but
% are scaled down if they would overflow the margins.
\makeatletter
\def\maxwidth{\ifdim\Gin@nat@width>\linewidth\linewidth
\else\Gin@nat@width\fi}
\makeatother
\let\Oldincludegraphics\includegraphics
\renewcommand{\includegraphics}[1]{\Oldincludegraphics[width=\maxwidth]{#1}}
$endif$


%  This code to use {caption} will wipe out the ability of pandoc-crossref
%  to prepend the "Figure" reference to the caption.
% \usepackage{caption}
% \DeclareCaptionLabelFormat{nolabel}{}
% \captionsetup{labelformat=nolabel}


\ifxetex
  \usepackage[setpagesize=false, % page size defined by xetex
              unicode=false, % unicode breaks when used with xetex
              xetex]{hyperref}
\else
  \usepackage[unicode=true]{hyperref}
\fi

\hypersetup{breaklinks=true, pdfborder={0 0 0}, colorlinks=true, urlcolor=bluehaze, linkcolor=bluehaze, citecolor=bluehaze, anchorcolor=bluehaze}
$if(strikeout)$
\usepackage[normalem]{ulem}
% avoid problems with \sout in headers with hyperref:
\pdfstringdefDisableCommands{\renewcommand{\sout}{}}
$endif$
$if(subscript)$
\newcommand{\textsubscr}[1]{\ensuremath{_{\scriptsize\textrm{#1}}}}
$endif$
\setlength{\parindent}{0pt}
\setlength{\parskip}{6pt plus 2pt minus 1pt}
\setlength{\emergencystretch}{3em}  % prevent overfull lines
$if(listings)$
\usepackage{listings}
$endif$
$if(numbersections)$
$else$
\setcounter{secnumdepth}{0}
$endif$
$if(verbatim-in-note)$
\VerbatimFootnotes % allows verbatim text in footnotes
$endif$
$for(header-includes)$
$header-includes$
$endfor$

$if(title)$
\title{$title$}
$endif$
% $if(author)$
% \author{$for(author)$$author$$sep$ \and $endfor$}
% $endif$
$if(date)$
\date{$date$}
$endif$



% my page number
%\pagestyle{fancy}
%\fancyhead[RO]{Sheffield \thepage}
\fancyhead{} % empty
\fancyfoot{} % empty out the default, which is page number in center
\def\middot{\textperiodcentered~}
\lfoot{\scriptsize \thepage \middot $short_title$ \middot \textit{Databio} \middot $if(date)$ \textsl{$date$} $endif$ \textcopyright The Authors}
\usepackage{fancyhdr}
\pagestyle{fancy}
\usepackage{titlesec}



\usepackage{mdframed}   % for framing

\newmdenv[linecolor=white, fontcolor=bluetint, backgroundcolor=graybit, leftmargin=8, rightmargin=8, innertopmargin=8, innerbottommargin=8, innerleftmargin=12, innerrightmargin=12, roundcorner=10pt, font=\large]{infobox}

%\usepackage{sectsty}% http://ctan.org/pkg/sectsty
\usepackage{xcolor}% http://ctan.org/pkg/xcolor
% \sectionfont{\color{bluehaze}\normalfont\fontsize{12}{3}\selectfont}
\linespread{1.0}

\titleformat{\section}
  {\normalfont\fontfamily{phv}\fontsize{12pt}{12pt}\selectfont\color{bluetint}\bfseries}{\thesection}{1em}{}

\titleformat{\subsection}
  {\normalfont\fontfamily{phv}\fontsize{11pt}{11pt}\selectfont\color{black}\bfseries}{\thesubsection}{1em}{}

\titleformat{\subsubsection}
  {\normalfont\fontfamily{phv}\fontsize{11pt}{11pt}\selectfont\color{black}\bfseries}{\thesubsubsection}{1em}{}


% \titlespacing{command}{left spacing}{before spacing}{after spacing}[right]
\titlespacing\section{0pt}{12pt plus 2pt minus 1pt}{-2pt plus 1pt minus 1pt}
\titlespacing\subsection{0pt}{11pt plus 2pt minus 1pt}{-2pt plus 1pt minus 1pt}
\titlespacing\subsubsection{0pt}{0pt plus 2pt minus 1pt}{-2pt plus 2pt minus 2pt}


$for(author)$
$if(author.name)$
  $if(author.correspondence)$
    \author[$author.affiliation$,*]{$author.name$}
  $else$
    \author[$author.affiliation$]{$author.name$}
  $endif$
$else$
\author{$author$}
$endif$
$endfor$

% institutions must come after \author definitions so that corresponding tag comes last.
$for(institutions)$
$if(institutions.name)$
\affil[$institutions.key$]{$institutions.name$}
$endif$
$endfor$

$for(author)$
  $if(author.correspondence)$
    \affil[*]{Correspondence: \href{mailto:$author.correspondence$}{$author.correspondence$}}
$endif$
$endfor$


% bring affils a little closer to author.
\setlength{\affilsep}{0.5em}

\usepackage{titling}
\renewcommand{\headrulewidth}{0pt}
\renewcommand{\Affilfont}{\scriptsize}
\makeatletter
\renewcommand{\maketitle}{%

$if(class)$
  \begin{flushright}{\huge{\color{graytint}\textbf{\MakeUppercase{$class$}}}}\end{flushright}\vspace{-0.5em}
  {\color{graytint}\hrule}
  $endif$
  \begin{raggedright} 
  \textbf{\huge{\@title}}
        \par \vspace{.5ex}
        % $for(author)$$author$$sep$ \and $endfor$
      {\ignorespaces\@author\newline\href{http://databio.org$permalink$}{http://databio.org$permalink$}\par}
       % \hrule
         \vspace{1ex}
        \end{raggedright}            
}   
\makeatother

%\subsectionfont{\color{maroonhaze}}
%\subsubsectionfont{\color{orangehaze}}
% end colors

% this code uses titles.sty to make a pagebreak with every header 1
% so each section will begin on a new page.
%\newcommand{\sectionbreak}{\clearpage}


\graphicspath{{/home/nsheff/code/databio.github.io/}}

% Redefine the figure environment so there's no float (what pandoc does by default)
% to make it compatible with the multicol package
\usepackage{float}
\let\origfigure\figure
\let\endorigfigure\endfigure
\renewenvironment{figure}[1][2] {
    \expandafter\origfigure\expandafter[H]
} {
    \endorigfigure
}



\newenvironment{widefig}{\renewenvironment{figure}{\begin{figure*}\centering}{\end{figure*}}}

% These two packages are used to write algorithms
\usepackage{algorithm}
\usepackage{algpseudocode}


\begin{document}


$if(title)$
\maketitle
$endif$

$if(abstract)$
\begin{infobox}
$abstract$
\end{infobox}
$endif$



$for(include-before)$
$include-before$

$endfor$
$if(toc)$
\tableofcontents

$endif$

\begin{multicols}{2}
$body$
\end{multicols}

$if(natbib)$
$if(biblio-files)$
$if(biblio-title)$
$if(book-class)$
\renewcommand\bibname{$biblio-title$}
$else$
\renewcommand\refname{$biblio-title$}
$endif$
$endif$
\bibliography{$biblio-files$}

$endif$
$endif$
$if(biblatex)$
\printbibliography$if(biblio-title)$[title=$biblio-title$]$endif$

$endif$
$for(include-after)$
$include-after$

$endfor$



\end{document}
%\documentclass$if(fontsize)$[$fontsize$]$endif${article}
\documentclass[12pt]{article}
\usepackage{amssymb,amsmath}
\usepackage{ifxetex,ifluatex}
\ifxetex
  \usepackage{fontspec,xltxtra,xunicode}
  \defaultfontfeatures{Mapping=tex-text,Scale=MatchLowercase}
\else
  \ifluatex
    \usepackage{fontspec}
    \defaultfontfeatures{Mapping=tex-text,Scale=MatchLowercase}
  \else
    \usepackage[utf8]{inputenc}
  \fi
\fi
$if(natbib)$
\usepackage{natbib}
\bibliographystyle{plainnat}
$endif$
$if(biblatex)$
\usepackage{biblatex}
$if(biblio-files)$
\bibliography{$biblio-files$}
$endif$
$endif$
$if(lhs)$
\usepackage{listings}
\lstnewenvironment{code}{\lstset{language=Haskell,basicstyle=\small\ttfamily}}{}
$endif$
$if(verbatim-in-note)$
\usepackage{fancyvrb}
$endif$
$if(fancy-enums)$
% Redefine labelwidth for lists; otherwise, the enumerate package will cause
% markers to extend beyond the left margin.
\makeatletter\AtBeginDocument{%
  \renewcommand{\@listi}
    {\setlength{\labelwidth}{4em}}
}\makeatother
\usepackage{enumerate}
$endif$
$if(tables)$
\usepackage{ctable}
\usepackage{float} % provides the H option for float placement
$endif$
$if(url)$
\usepackage{url}
$endif$
$if(graphics)$
\usepackage{graphicx}
% We will generate all images so they have a width \maxwidth. This means
% that they will get their normal width if they fit onto the page, but
% are scaled down if they would overflow the margins.
\makeatletter
\def\maxwidth{\ifdim\Gin@nat@width>\linewidth\linewidth
\else\Gin@nat@width\fi}
\makeatother
\let\Oldincludegraphics\includegraphics
\renewcommand{\includegraphics}[1]{\Oldincludegraphics[width=\maxwidth]{#1}}
$endif$


%  This code to use {caption} will wipe out the ability of pandoc-crossref
%  to prepend the "Figure" reference to the caption.
% \usepackage{caption}
% \DeclareCaptionLabelFormat{nolabel}{}
% \captionsetup{labelformat=nolabel}



\ifxetex
  \usepackage[setpagesize=false, % page size defined by xetex
              unicode=false, % unicode breaks when used with xetex
              xetex]{hyperref}
\else
  \usepackage[unicode=true]{hyperref}
\fi
\hypersetup{breaklinks=true, pdfborder={0 0 0}}
$if(strikeout)$
\usepackage[normalem]{ulem}
% avoid problems with \sout in headers with hyperref:
\pdfstringdefDisableCommands{\renewcommand{\sout}{}}
$endif$
$if(subscript)$
\newcommand{\textsubscr}[1]{\ensuremath{_{\scriptsize\textrm{#1}}}}
$endif$
\setlength{\parindent}{0pt}
\setlength{\parskip}{6pt plus 2pt minus 1pt}
\setlength{\emergencystretch}{3em}  % prevent overfull lines
$if(listings)$
\usepackage{listings}
$endif$
$if(numbersections)$
$else$
\setcounter{secnumdepth}{0}
$endif$
$if(verbatim-in-note)$
\VerbatimFootnotes % allows verbatim text in footnotes
$endif$
$for(header-includes)$
$header-includes$
$endfor$

$if(title)$
\title{$title$}
$endif$
% $if(author)$
% \author{$for(author)$$author$$sep$ \and $endfor$}
% $endif$
$if(date)$
\date{$date$}
$endif$

\begin{document}
$if(title)$
\maketitle
$endif$

$for(include-before)$
$include-before$

$endfor$
$if(toc)$
\tableofcontents

$endif$
$body$

$if(natbib)$
$if(biblio-files)$
$if(biblio-title)$
$if(book-class)$
\renewcommand\bibname{$biblio-title$}
$else$
\renewcommand\refname{$biblio-title$}
$endif$
$endif$
\bibliography{$biblio-files$}

$endif$
$endif$
$if(biblatex)$
\printbibliography$if(biblio-title)$[title=$biblio-title$]$endif$

$endif$
$for(include-after)$
$include-after$

$endfor$
\end{document}


