% \renewcommand{\rmdefault}{phv} % Arial
% \renewcommand{\sfdefault}{phv} % Arial

\renewcommand{\rmdefault}{bch} % Charter
\renewcommand{\sfdefault}{bch} % Charter

\documentclass$if(fontsize)$[$fontsize$]$else$[10pt]$endif${article}
\RequirePackage[top=0.5in,left=0.5in,right=0.5in,bottom=0.5in]{geometry}

\usepackage{bold-extra}
\usepackage{pdfpages}

\usepackage{wrapfig}

% longtable is required for markdown-format tables to be rendered in latex
\usepackage{longtable}


% my page number
\usepackage{fancyhdr}
\pagestyle{fancy}

\fancyhead{} 

% \fancyhead{} % empty
\fancyfoot{} % empty out the default, which is page number in center
\fancyfoot[R]{\footnotesize Revised \today}
\fancyfoot[C]{\footnotesize Sheffield \thepage}

% Handle first page (not otherwise affectecd by fancyfoot etc.)
\fancypagestyle{plain}{%
  \renewcommand{\headrulewidth}{0pt}%
  \fancyhf{}%
  \fancyfoot{}
}



\usepackage{amssymb,amsmath}
\usepackage{ifxetex,ifluatex}
\ifxetex
  \usepackage{fontspec,xltxtra,xunicode}
  \defaultfontfeatures{Mapping=tex-text,Scale=MatchLowercase}
\else
  \ifluatex
    \usepackage{fontspec}
    \defaultfontfeatures{Mapping=tex-text,Scale=MatchLowercase}
  \else
    \usepackage[utf8]{inputenc}
  \fi
\fi
$if(natbib)$
\usepackage{natbib}
\bibliographystyle{plainnat}
$endif$
$if(biblatex)$
\usepackage{biblatex}
$if(biblio-files)$
\bibliography{$biblio-files$}
$endif$
$endif$
$if(lhs)$
\usepackage{listings}
\lstnewenvironment{code}{\lstset{language=Haskell,basicstyle=\small\ttfamily}}{}
$endif$
$if(verbatim-in-note)$
\usepackage{fancyvrb}
$endif$
$if(fancy-enums)$
% Redefine labelwidth for lists; otherwise, the enumerate package will cause
% markers to extend beyond the left margin.
\makeatletter\AtBeginDocument{%
  \renewcommand{\@listi}
    {\setlength{\labelwidth}{4em}}
}\makeatother
\usepackage{enumerate}

% Attempt to smash lists together
\setlist[enumerate]{itemsep=1mm}

$endif$
$if(tables)$
\usepackage{ctable}
\usepackage{float} % provides the H option for float placement
$endif$
$if(url)$
\usepackage{url}
$endif$
$if(graphics)$
\usepackage{graphicx}
% We will generate all images so they have a width \maxwidth. This means
% that they will get their normal width if they fit onto the page, but
% are scaled down if they would overflow the margins.
\makeatletter
\def\maxwidth{\ifdim\Gin@nat@width>\linewidth\linewidth
\else\Gin@nat@width\fi}
\makeatother
\let\Oldincludegraphics\includegraphics
\renewcommand{\includegraphics}[1]{\Oldincludegraphics[width=\maxwidth]{#1}}
$endif$
\usepackage{caption}
\DeclareCaptionLabelFormat{nolabel}{}
\captionsetup{labelformat=nolabel}
\ifxetex
  \usepackage[setpagesize=false, % page size defined by xetex
              unicode=false, % unicode breaks when used with xetex
              xetex]{hyperref}
\else
  \usepackage[unicode=true]{hyperref}
\fi
\hypersetup{breaklinks=true, pdfborder={0 0 0}}
$if(strikeout)$
\usepackage[normalem]{ulem}
% avoid problems with \sout in headers with hyperref:
\pdfstringdefDisableCommands{\renewcommand{\sout}{}}
$endif$
$if(subscript)$
\newcommand{\textsubscr}[1]{\ensuremath{_{\scriptsize\textrm{#1}}}}
$endif$
\setlength{\parindent}{0pt}
\setlength{\parskip}{12pt plus 2pt minus 1pt}
\setlength{\emergencystretch}{3em}  % prevent overfull lines
$if(listings)$
\usepackage{listings}
$endif$
$if(numbersections)$
$else$
\setcounter{secnumdepth}{0}
$endif$
$if(verbatim-in-note)$
\VerbatimFootnotes % allows verbatim text in footnotes
$endif$
$for(header-includes)$
$header-includes$
$endfor$

$if(title)$
\title{$title$}
$endif$
$if(author)$
\author{$for(author)$$author$$sep$ \and $endfor$}
$endif$
$if(date)$
\date{$date$}
$endif$

\renewcommand{\headrulewidth}{0pt}

% tabto is required to use /tab control sequence.
% We use the tabto package to set tabs for the Research support section
% This sets the tab stop positions.
\usepackage{tabto}
\TabPositions{2.5cm, 9.25cm,14.25cm}


\usepackage{titlesec}


% colors
\definecolor{bluehaze}{RGB}{54,95,145}
\definecolor{maroonhaze}{RGB}{148,54,52}
\definecolor{orangehaze}{RGB}{227,108,10}
\definecolor{bluetint}{RGB}{0,35,75}
\definecolor{bluetint2}{RGB}{0,35,70}
\definecolor{lightblack}{RGB}{75,75,75}
%\usepackage{sectsty}% http://ctan.org/pkg/sectsty
\usepackage{xcolor}% http://ctan.org/pkg/xcolor
% \sectionfont{\color{bluehaze}\normalfont\fontsize{12}{3}\selectfont}

% make hyperlinks colorful
\hypersetup{
  colorlinks=true,
  linkcolor=bluehaze,
  urlcolor=bluehaze
}


% Spacing

 \let\tightlist\relax
% \def\labelitemi{\raisebox{0.5ex}{\tiny\textbullet}}
% this itemindent code adjusts the indentation of bullet lists.

\providecommand{\tightlist}{%
  \setlength{\itemsep}{2pt}\setlength{\parskip}{0pt}\addtolength{\itemindent}{-2pt} 
}


\usepackage{needspace}
$if(titleneedspace)$
\def \titleneedspace{$titleneedspace$}
$else$
\def \titleneedspace{0cm}
$endif$
\linespread{1.0}
% \titleformat{<command>}[<shape>]{<format>}{<label>}{<sec>}{<before-code>}[<after-code>]
% \fontsize{<size>}{<bskip>} sets respectively the font size and the baseline skip.

\titleformat{\section}
  {\needspace{\titleneedspace}\normalfont\fontfamily{bch}\fontsize{12pt}{2pt}\selectfont\color{black}\bfseries\scshape}{\thesection}{1em}{}

\titleformat{\subsection}
  {\needspace{\titleneedspace}\normalfont\fontfamily{bch}\fontsize{10pt}{8pt}\selectfont\color{black}\bfseries}{\makebox[2in][l]\thesubsection}{1em}{}

\titleformat{\subsubsection}
  {\needspace{\titleneedspace}\normalfont\fontfamily{bch}\fontsize{9pt}{8pt}\selectfont\color{bluetint}\itshape}{\thesubsubsection}{1em}{}




\usepackage{amsthm}
\makeatletter
\newcommand{\addperiod}[1]{#1\@addpunct{.}}
\makeatother
% use runin to specify no newline
\titleformat{\paragraph}[runin]
  {\normalfont\fontfamily{phv}\fontsize{9pt}{8pt}\selectfont\color{lightblack}\itshape}{\theparagraph}{}{\addperiod}




% \titlespacing{command}{left spacing}{before spacing}{after spacing}[right]
% \titlespacing\section{0pt}{4pt plus 4pt minus 4pt}{0pt plus 4pt minus 4pt}
% \titlespacing\subsection{8pt}{0pt plus 2pt minus 2pt}{0pt plus 2pt minus 2pt}
% \titlespacing\subsubsection{16pt}{0pt plus 2pt minus 2pt}{0pt plus 2pt minus 2pt}

\titlespacing\section{0pt}{2pt plus 2pt minus 2pt}{-10pt plus 2pt minus 2pt}
\titlespacing\subsection{0pt}{9pt plus 2pt minus 2pt}{-1pt plus 2pt minus 2pt}
\titlespacing\subsubsection{0pt}{0pt plus 4pt minus 2pt}{-2pt plus 2pt minus 2pt}
\titlespacing{\paragraph}{0pt}{1pt plus 4pt minus 2pt}{2pt plus 2pt minus 1pt}


% This chunk removes some of the space and reduces size of the "title" name

\usepackage{titling}
\pretitle{\begin{center}\large}
\posttitle{\end{center}\vspace{-3ex}}
\preauthor{\begin{center}\fontsize{10pt}{10pt}}
\postauthor{\end{center}\vspace{-3ex}}
\predate{\begin{center}}
\postdate{\end{center}\vspace{-3ex}}
\setlength{\droptitle}{-40pt}


%\subsectionfont{\color{maroonhaze}}
%\subsubsectionfont{\color{orangehaze}}
% end colors

% this code uses titles.sty to make a pagebreak with every header 1
% so each section will begin on a new page.
%\newcommand{\sectionbreak}{\clearpage}


% #\renewcommand{\labelitemi}{\tiny\textbullet} # this doesn't work right; needs raisebox



\makeatletter
    \def\@maketitle{%
  \newpage
  \null
  % \vskip 2em%
  \vspace{-8ex}
  \begin{center}%
  \let \footnote \thanks
    \vskip 2pt
    \bfseries{\textsc{\large\@title - \@author}}\par
    \vspace{-2ex}\@date
  \end{center}%
  \par
  % \vskip 1.5em
  }
\makeatother




% for pandoc 2.11
$if(csl-refs)$
\newlength{\cslhangindent}
\setlength{\cslhangindent}{1.5em}
\newlength{\csllabelwidth}
\setlength{\csllabelwidth}{3em}
\newenvironment{CSLReferences}[2] % #1 hanging-ident, #2 entry spacing
 {% don't indent paragraphs
  \setlength{\parindent}{0pt}
  % turn on hanging indent if param 1 is 1
  \ifodd #1 \everypar{\setlength{\hangindent}{\cslhangindent}}\ignorespaces\fi
  % set entry spacing
  \ifnum #2 > 0
  \setlength{\parskip}{#2\baselineskip}
  \fi
 }%
 {}
\usepackage{calc}
\newcommand{\CSLBlock}[1]{#1\hfill\break}
\newcommand{\CSLLeftMargin}[1]{\parbox[t]{\csllabelwidth}{#1}}
\newcommand{\CSLRightInline}[1]{\parbox[t]{\linewidth - \csllabelwidth}{#1}\break}
\newcommand{\CSLIndent}[1]{\hspace{\cslhangindent}#1}
$endif$



\def\labelitemi{\raisebox{0.5ex}{\tiny\textbullet}}


\begin{document}

$if(title)$
\maketitle
$endif$

$for(include-before)$
$include-before$

$endfor$
$if(toc)$
\tableofcontents

$endif$
$body$

$if(natbib)$
$if(biblio-files)$
$if(biblio-title)$
$if(book-class)$
\renewcommand\bibname{$biblio-title$}
$else$
\renewcommand\refname{$biblio-title$}
$endif$
$endif$
\bibliography{$biblio-files$}

$endif$
$endif$
$if(biblatex)$
\printbibliography$if(biblio-title)$[title=$biblio-title$]$endif$

$endif$
$for(include-after)$
$include-after$

$endfor$



\end{document}
%\documentclass$if(fontsize)$[$fontsize$]$endif${article}
\documentclass[10pt]{article}
\usepackage{amssymb,amsmath}
\usepackage{ifxetex,ifluatex}
\ifxetex
  \usepackage{fontspec,xltxtra,xunicode}
  \defaultfontfeatures{Mapping=tex-text,Scale=MatchLowercase}
\else
  \ifluatex
    \usepackage{fontspec}
    \defaultfontfeatures{Mapping=tex-text,Scale=MatchLowercase}
  \else
    \usepackage[utf8]{inputenc}
  \fi
\fi
$if(natbib)$
\usepackage{natbib}
\bibliographystyle{plainnat}
$endif$
$if(biblatex)$
\usepackage{biblatex}
$if(biblio-files)$
\bibliography{$biblio-files$}
$endif$
$endif$
$if(lhs)$
\usepackage{listings}
\lstnewenvironment{code}{\lstset{language=Haskell,basicstyle=\small\ttfamily}}{}
$endif$
$if(verbatim-in-note)$
\usepackage{fancyvrb}
$endif$
$if(fancy-enums)$
% Redefine labelwidth for lists; otherwise, the enumerate package will cause
% markers to extend beyond the left margin.
\makeatletter\AtBeginDocument{%
  \renewcommand{\@listi}
    {\setlength{\labelwidth}{4em}}
}\makeatother
\usepackage{enumerate}
$endif$
$if(tables)$
\usepackage{ctable}
\usepackage{float} % provides the H option for float placement
$endif$
$if(url)$
\usepackage{url}
$endif$
$if(graphics)$
\usepackage{graphicx}
% We will generate all images so they have a width \maxwidth. This means
% that they will get their normal width if they fit onto the page, but
% are scaled down if they would overflow the margins.
\makeatletter
\def\maxwidth{\ifdim\Gin@nat@width>\linewidth\linewidth
\else\Gin@nat@width\fi}
\makeatother
\let\Oldincludegraphics\includegraphics
\renewcommand{\includegraphics}[1]{\Oldincludegraphics[width=\maxwidth]{#1}}
$endif$
\usepackage{caption}
\DeclareCaptionLabelFormat{nolabel}{}
\captionsetup{labelformat=nolabel}
\ifxetex
  \usepackage[setpagesize=false, % page size defined by xetex
              unicode=false, % unicode breaks when used with xetex
              xetex]{hyperref}
\else
  \usepackage[unicode=true]{hyperref}
\fi
\hypersetup{breaklinks=true, pdfborder={0 0 0}}
$if(strikeout)$
\usepackage[normalem]{ulem}
% avoid problems with \sout in headers with hyperref:
\pdfstringdefDisableCommands{\renewcommand{\sout}{}}
$endif$
$if(subscript)$
\newcommand{\textsubscr}[1]{\ensuremath{_{\scriptsize\textrm{#1}}}}
$endif$
\setlength{\parindent}{0pt}
\setlength{\parskip}{6pt plus 2pt minus 1pt}
\setlength{\emergencystretch}{3em}  % prevent overfull lines
$if(listings)$
\usepackage{listings}
$endif$
$if(numbersections)$
$else$
\setcounter{secnumdepth}{0}
$endif$
$if(verbatim-in-note)$
\VerbatimFootnotes % allows verbatim text in footnotes
$endif$
$for(header-includes)$
$header-includes$
$endfor$

$if(title)$
\title{$title$}
$endif$
$if(author)$
\author{$for(author)$$author$$sep$ \and $endfor$}
$endif$
$if(date)$
\date{$date$}
$endif$

\begin{document}
$if(title)$
\maketitle
$endif$

$for(include-before)$
$include-before$

$endfor$
$if(toc)$
\tableofcontents

$endif$
$body$

$if(natbib)$
$if(biblio-files)$
$if(biblio-title)$
$if(book-class)$
\renewcommand\bibname{$biblio-title$}
$else$
\renewcommand\refname{$biblio-title$}
$endif$
$endif$
\bibliography{$biblio-files$}

$endif$
$endif$
$if(biblatex)$
\printbibliography$if(biblio-title)$[title=$biblio-title$]$endif$

$endif$
$for(include-after)$
$include-after$

$endfor$
\end{document}
