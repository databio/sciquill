% This latex/pandoc NIH biosketch template has borrowed components from pmagwene's
% latex version of the NIH biosketch.
% Thanks to pmagwene for sharing and inspiring this rendition

\renewcommand{\rmdefault}{phv} % Arial
\renewcommand{\sfdefault}{phv} % Arial

 \let\tightlist\relax

\documentclass[11pt]{article}
\RequirePackage[top=0.5in,left=0.5in,right=0.5in,bottom=0.5in]{geometry}

\usepackage{pdfpages}
\usepackage{fancyhdr}
\RequirePackage{tabu}
\usepackage{wrapfig}
\usepackage[skip=2pt, font=small, format=plain, labelfont=it, textfont=it, tableposition=below]{caption}

% longtable is required for markdown-format tables to be rendered in latex
\usepackage{longtable}

\usepackage{amssymb,amsmath}
\usepackage{ifxetex,ifluatex}
\ifxetex
  \usepackage{fontspec,xltxtra,xunicode}
  \defaultfontfeatures{Mapping=tex-text,Scale=MatchLowercase}
\else
  \ifluatex
    \usepackage{fontspec}
    \defaultfontfeatures{Mapping=tex-text,Scale=MatchLowercase}
  \else
    \usepackage[utf8]{inputenc}
  \fi
\fi


% We use the tabto package to set tabs for the Research support section
% This sets the tab stop positions.
\usepackage{tabto}
\TabPositions{10.5cm,15.25cm}

\definecolor{bluehaze}{RGB}{54,95,145}
\definecolor{maroonhaze}{RGB}{148,54,52}
\definecolor{orangehaze}{RGB}{227,108,10}
\definecolor{black}{RGB}{0,0,0}
 
$if(natbib)$
\usepackage{natbib}
\bibliographystyle{plainnat}
$endif$
$if(biblatex)$
\usepackage{biblatex}
$if(biblio-files)$
\bibliography{$biblio-files$}
$endif$
$endif$
$if(lhs)$
\usepackage{listings}
\lstnewenvironment{code}{\lstset{language=Haskell,basicstyle=\small\ttfamily}}{}
$endif$
$if(verbatim-in-note)$
\usepackage{fancyvrb}
$endif$
$if(fancy-enums)$
% Redefine labelwidth for lists; otherwise, the enumerate package will cause
% markers to extend beyond the left margin.
\makeatletter\AtBeginDocument{%
  \renewcommand{\@listi}
    {\setlength{\labelwidth}{4em}}
}\makeatother
\usepackage{enumerate}
$endif$
$if(tables)$
\usepackage{ctable}

\usepackage{float} % provides the H option for float placement

$endif$

$if(url)$
\usepackage{url}
$endif$
$if(graphics)$
\usepackage{graphicx}
% We will generate all images so they have a width \maxwidth. This means
% that they will get their normal width if they fit onto the page, but
% are scaled down if they would overflow the margins.
\makeatletter
\def\maxwidth{\ifdim\Gin@nat@width>\linewidth\linewidth
\else\Gin@nat@width\fi}
\makeatother
\let\Oldincludegraphics\includegraphics
\renewcommand{\includegraphics}[1]{\Oldincludegraphics[width=\maxwidth]{#1}}
$endif$
\usepackage{caption}
\DeclareCaptionLabelFormat{nolabel}{}
\captionsetup{labelformat=nolabel}

% NIH does not allow hyperlinks in biosketch. This turns them off.
% \ifxetex
%   \usepackage[setpagesize=false, % page size defined by xetex
%               unicode=false, % unicode breaks when used with xetex
%               xetex]{hyperref}
% \else
%   \usepackage[unicode=true]{hyperref}
% \fi

% link color
% \hypersetup{breaklinks=true, pdfborder={0 0 0}, colorlinks=true, urlcolor=bluehaze}
% NIH does not allow hyperlinks in biosketch. This turns them off.
% I commented out the above and use this instead.
\usepackage{nohyperref}
\usepackage{url}


$if(strikeout)$
\usepackage[normalem]{ulem}
% avoid problems with \sout in headers with hyperref:
\pdfstringdefDisableCommands{\renewcommand{\sout}{}}
$endif$
$if(subscript)$
\newcommand{\textsubscr}[1]{\ensuremath{_{\scriptsize\textrm{#1}}}}
$endif$
\setlength{\parindent}{0pt}
% Controls distance between paragraphs
\setlength{\parskip}{6pt plus 4pt minus 4pt}
\setlength{\emergencystretch}{3em}  % prevent overfull lines
$if(listings)$
\usepackage{listings}
$endif$
$if(numbersections)$
$else$
\setcounter{secnumdepth}{0}
$endif$
$if(verbatim-in-note)$
\VerbatimFootnotes % allows verbatim text in footnotes
$endif$
$for(header-includes)$
$header-includes$
$endfor$

$if(title)$
\title{$title$}
$endif$
$if(author)$
\author{$for(author)$$author$$sep$ \and $endfor$}
\name{$for(author)$$author$$sep$ \and $endfor$}
$endif$
$if(date)$
\date{$date$}
$endif$

% for pandoc 2.11
$if(csl-refs)$
\newlength{\cslhangindent}
\setlength{\cslhangindent}{1.5em}
\newlength{\csllabelwidth}
\setlength{\csllabelwidth}{3em}
\newenvironment{CSLReferences}[2] % #1 hanging-ident, #2 entry spacing
 {% don't indent paragraphs
  \setlength{\parindent}{0pt}
  % turn on hanging indent if param 1 is 1
  \ifodd #1 \everypar{\setlength{\hangindent}{\cslhangindent}}\ignorespaces\fi
  % set entry spacing
  \ifnum #2 > 0
  \setlength{\parskip}{#2\baselineskip}
  \fi
 }%
 {}
\usepackage{calc}
\newcommand{\CSLBlock}[1]{#1\hfill\break}
\newcommand{\CSLLeftMargin}[1]{\parbox[t]{\csllabelwidth}{#1}}
\newcommand{\CSLRightInline}[1]{\parbox[t]{\linewidth - \csllabelwidth}{#1}\break}
\newcommand{\CSLIndent}[1]{\hspace{\cslhangindent}#1}
$endif$




% my page number
%\pagestyle{fancy}
%\fancyhead[RO]{Sheffield \thepage}
\fancyhead{} % empty
\fancyfoot{} % empty out the default, which is page number in center
\renewcommand{\headrulewidth}{0pt}
\pagenumbering{gobble}

\usepackage[nobottomtitles*]{titlesec}




% colors

%\usepackage{sectsty}% http://ctan.org/pkg/sectsty
\usepackage{xcolor}% http://ctan.org/pkg/xcolor
% \sectionfont{\color{bluehaze}\normalfont\fontsize{12}{3}\selectfont}
\linespread{1.0}

\titleformat{\section}
  {\normalfont\fontfamily{phv}\fontsize{11pt}{11pt}\selectfont\color{black}\bfseries}{\thesection}{1em}{}

\titleformat{\subsection}
  {\normalfont\fontfamily{phv}\fontsize{11pt}{11pt}\selectfont\color{black}\itshape}{\thesubsection}{1em}{}

\titleformat{\subsubsection}
  {\normalfont\fontfamily{phv}\fontsize{11pt}{11pt}\selectfont\color{black}\itshape}{\thesubsubsection}{1em}{}


% \titlespacing{command}{left spacing}{before spacing}{after spacing}[right]
\titlespacing\section{0pt}{10pt plus 4pt minus 6pt}{-2pt plus 3pt minus 3pt}
\titlespacing\subsection{0pt}{10pt plus 4pt minus 6pt}{-2pt plus 3pt minus 3pt}
\titlespacing\subsubsection{0pt}{10pt plus 4pt minus 6pt}{-4pt plus 4pt minus 4pt}


%\subsectionfont{\color{maroonhaze}}
%\subsubsectionfont{\color{orangehaze}}
% end colors

% this code uses titles.sty to make a pagebreak with every header 1
% so each section will begin on a new page.
%\newcommand{\sectionbreak}{\clearpage}
\newcommand{\pgline}{\vspace{-6pt}\noindent\makebox[\linewidth]{\rule{\textwidth}{0.6pt}}\newline}


\newcommand{\piinfo}
{\begin{footnotesize}\begin{flushright}OMB No. 0925-0001 and 0925-0002 (Rev. 03/2020 Approved Through 02/28/2023)\end{flushright}\end{footnotesize}\vspace{-1em}\pgline
{\centering
\textbf{BIOGRAPHICAL SKETCH}\\
{\footnotesize
Provide the following information for the Senior/key personnel and other significant contributors.\\
Follow this format for each person.  \textbf{DO NOT EXCEED FIVE PAGES}.\\}}
\pgline
NAME: $name$\vspace{0.5em}\\
eRA COMMONS USER NAME (credential, e.g., agency login): $eracommons$\vspace{0.5em}\\
POSITION TITLE: $position$\vspace{0.5em}\\
EDUCATION/TRAINING (\textit{Begin with baccalaureate or other initial professional education, such as nursing, include postdoctoral training and residency training if applicable. Add/delete rows as necessary.})\vspace{0.5em}\\}

\newenvironment{education}
{%
\bgroup\centering
\tabulinesep=1.5mm
\begin{tabu} to \textwidth {X[3.53,l,m]|X[1,c,m]|X[1,c,m]|X[2.15,l,m]}
\hline
\rowfont[c]{}               % for the header only we want all the text centered
INSTITUTION AND LOCATION &
DEGREE & 
Completion Date & 
FIELD OF STUDY\\
\hline}
{\end{tabu}\egroup}

% make bullets smaller
\def\labelitemi{}
% this itemindent code adjusts the indentation of bullet lists.
\providecommand{\tightlist}{%
  \setlength{\itemsep}{0pt}\setlength{\parskip}{0pt}\addtolength{\itemindent}{-24pt}
}

\usepackage{enumitem}
% Attempt to smash lists together
\setlist[enumerate]{wide, itemsep=0.5mm, labelindent=0pt, leftmargin=0pt, labelwidth=!, itemindent=0pt, font=\itshape}

\setlist[enumerate]{partopsep=30pt,topsep=30pt}

\begin{document}

$if(title)$
\maketitle
$endif$
$if(piinfo)$
\piinfo
$endif$
% \begin{education}
% $education$
% \end{education}
$for(include-before)$
$include-before$

$endfor$
$if(toc)$
\tableofcontents

$endif$
$body$

$if(natbib)$
$if(biblio-files)$
$if(biblio-title)$
$if(book-class)$
\renewcommand\bibname{$biblio-title$}
$else$
\renewcommand\refname{$biblio-title$}
$endif$
$endif$
\bibliography{$biblio-files$}

$endif$
$endif$
$if(biblatex)$
\printbibliography$if(biblio-title)$[title=$biblio-title$]$endif$

$endif$
$for(include-after)$
$include-after$

$endfor$



\end{document}
%\documentclass$if(fontsize)$[$fontsize$]$endif${article}
\documentclass[12pt]{article}
\usepackage{amssymb,amsmath}
\usepackage{ifxetex,ifluatex}
\ifxetex
  \usepackage{fontspec,xltxtra,xunicode}
  \defaultfontfeatures{Mapping=tex-text,Scale=MatchLowercase}
\else
  \ifluatex
    \usepackage{fontspec}
    \defaultfontfeatures{Mapping=tex-text,Scale=MatchLowercase}
  \else
    \usepackage[utf8]{inputenc}
  \fi
\fi
$if(natbib)$
\usepackage{natbib}
\bibliographystyle{plainnat}
$endif$
$if(biblatex)$
\usepackage{biblatex}
$if(biblio-files)$
\bibliography{$biblio-files$}
$endif$
$endif$
$if(lhs)$
\usepackage{listings}
\lstnewenvironment{code}{\lstset{language=Haskell,basicstyle=\small\ttfamily}}{}
$endif$
$if(verbatim-in-note)$
\usepackage{fancyvrb}
$endif$
$if(fancy-enums)$
% Redefine labelwidth for lists; otherwise, the enumerate package will cause
% markers to extend beyond the left margin.
\makeatletter\AtBeginDocument{%
  \renewcommand{\@listi}
    {\setlength{\labelwidth}{4em}}
}\makeatother
\usepackage{enumerate}
$endif$
$if(tables)$
\usepackage{ctable}
\usepackage{float} % provides the H option for float placement
$endif$
$if(url)$
\usepackage{url}
$endif$
$if(graphics)$
\usepackage{graphicx}
% We will generate all images so they have a width \maxwidth. This means
% that they will get their normal width if they fit onto the page, but
% are scaled down if they would overflow the margins.
\makeatletter
\def\maxwidth{\ifdim\Gin@nat@width>\linewidth\linewidth
\else\Gin@nat@width\fi}
\makeatother
\let\Oldincludegraphics\includegraphics
\renewcommand{\includegraphics}[1]{\Oldincludegraphics[width=\maxwidth]{#1}}
$endif$
\usepackage{caption}
\DeclareCaptionLabelFormat{nolabel}{}
\captionsetup{labelformat=nolabel}
\ifxetex
  \usepackage[setpagesize=false, % page size defined by xetex
              unicode=false, % unicode breaks when used with xetex
              xetex]{hyperref}
\else
  \usepackage[unicode=true]{hyperref}
\fi
\hypersetup{breaklinks=true, pdfborder={0 0 0}}
$if(strikeout)$
\usepackage[normalem]{ulem}
% avoid problems with \sout in headers with hyperref:
\pdfstringdefDisableCommands{\renewcommand{\sout}{}}
$endif$
$if(subscript)$
\newcommand{\textsubscr}[1]{\ensuremath{_{\scriptsize\textrm{#1}}}}
$endif$
\setlength{\parindent}{0pt}
\setlength{\parskip}{6pt plus 2pt minus 1pt}
\setlength{\emergencystretch}{3em}  % prevent overfull lines
$if(listings)$
\usepackage{listings}
$endif$
$if(numbersections)$
$else$
\setcounter{secnumdepth}{0}
$endif$
$if(verbatim-in-note)$
\VerbatimFootnotes % allows verbatim text in footnotes
$endif$
$for(header-includes)$
$header-includes$
$endfor$

$if(title)$
\title{$title$}
$endif$
$if(author)$
\author{$for(author)$$author$$sep$ \and $endfor$}
$endif$
$if(date)$
\date{$date$}
$endif$

\begin{document}
$if(title)$
\maketitle
$endif$

$for(include-before)$
$include-before$

$endfor$
$if(toc)$
\tableofcontents

$endif$
$body$

$if(natbib)$
$if(biblio-files)$
$if(biblio-title)$
$if(book-class)$
\renewcommand\bibname{$biblio-title$}
$else$
\renewcommand\refname{$biblio-title$}
$endif$
$endif$
\bibliography{$biblio-files$}

$endif$
$endif$
$if(biblatex)$
\printbibliography$if(biblio-title)$[title=$biblio-title$]$endif$

$endif$
$for(include-after)$
$include-after$

$endfor$
\end{document}
