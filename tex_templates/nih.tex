\renewcommand{\rmdefault}{phv} % Arial
\renewcommand{\sfdefault}{phv} % Arial

 \let\tightlist\relax


\providecommand{\tightlist}{%
  \setlength{\itemsep}{0pt}\setlength{\parskip}{0pt}}

  
\documentclass[11pt]{article}
\RequirePackage[top=0.5in,left=0.5in,right=0.5in,bottom=0.5in]{geometry}

% hack for markdown processing. by default, text enclosed between latex \begin
% and \end tags will *not* be parsed by pandoc. If we *do* want to parse that
% code (because it's written in markdown), we can just add aliases for \Begin
% and \End, which are not flagged by pandoc as "avoid parsing". So markdown
% enclosed by \Begin and \End will be parsed as by pandoc.
% Use this to define a two-column reference sectiona t the end of your document
% like this:
% \Begin{multicols}{2}
% \tiny
% \setlength{\parskip}{0.2em}
% <div id="refs"></div>
% \End{multicols}


\let\Begin\begin
\let\End\end


\usepackage{pdfpages}
\usepackage{fancyhdr}
\usepackage{multicol}
\usepackage{wrapfig}
\usepackage[skip=2pt, font=small, format=plain, labelfont=it, textfont=it, tableposition=below]{caption}

% longtable is required for markdown-format tables to be rendered in latex
\usepackage{longtable}

\usepackage{amssymb,amsmath}
\usepackage{ifxetex,ifluatex}
\ifxetex
  \usepackage{fontspec,xltxtra,xunicode}
  \defaultfontfeatures{Mapping=tex-text,Scale=MatchLowercase}
\else
  \ifluatex
    \usepackage{fontspec}
    \defaultfontfeatures{Mapping=tex-text,Scale=MatchLowercase}
  \else
    \usepackage[utf8]{inputenc}
  \fi
\fi
$if(natbib)$
\usepackage{natbib}
\bibliographystyle{plainnat}
$endif$
$if(biblatex)$
\usepackage{biblatex}
$if(biblio-files)$
\bibliography{$biblio-files$}
$endif$
$endif$
$if(lhs)$
\usepackage{listings}
\lstnewenvironment{code}{\lstset{language=Haskell,basicstyle=\small\ttfamily}}{}
$endif$
$if(verbatim-in-note)$
\usepackage{fancyvrb}
$endif$
$if(fancy-enums)$
% Redefine labelwidth for lists; otherwise, the enumerate package will cause
% markers to extend beyond the left margin.
\makeatletter\AtBeginDocument{%
  \renewcommand{\@listi}
    {\setlength{\labelwidth}{4em}}
}\makeatother
\usepackage{enumerate}

% Attempt to smash lists together
\setlist[enumerate]{itemsep=1mm}

$endif$
$if(tables)$
\usepackage{ctable}
\usepackage{float} % provides the H option for float placement
$endif$
$if(url)$
\usepackage{url}
$endif$
$if(graphics)$
\usepackage{graphicx}
% We will generate all images so they have a width \maxwidth. This means
% that they will get their normal width if they fit onto the page, but
% are scaled down if they would overflow the margins.
\makeatletter
\def\maxwidth{\ifdim\Gin@nat@width>\linewidth\linewidth
\else\Gin@nat@width\fi}
\makeatother
\let\Oldincludegraphics\includegraphics
\renewcommand{\includegraphics}[1]{\Oldincludegraphics[width=\maxwidth]{#1}}
$endif$
\usepackage{caption}
\DeclareCaptionLabelFormat{nolabel}{}
\captionsetup{labelformat=nolabel}
\ifxetex
  \usepackage[setpagesize=false, % page size defined by xetex
              unicode=false, % unicode breaks when used with xetex
              xetex]{hyperref}
\else
  \usepackage[unicode=true]{hyperref}
\fi
\hypersetup{breaklinks=true, pdfborder={0 0 0}, colorlinks=true, urlcolor=bluehaze, linkcolor=bluehaze, citecolor=bluehaze, anchorcolor=bluehaze}
$if(strikeout)$
\usepackage[normalem]{ulem}
% avoid problems with \sout in headers with hyperref:
\pdfstringdefDisableCommands{\renewcommand{\sout}{}}
$endif$
$if(subscript)$
\newcommand{\textsubscr}[1]{\ensuremath{_{\scriptsize\textrm{#1}}}}
$endif$
\setlength{\parindent}{0pt}
\setlength{\parskip}{6pt plus 2pt minus 1pt}
\setlength{\emergencystretch}{3em}  % prevent overfull lines
$if(listings)$
\usepackage{listings}
$endif$
$if(numbersections)$
$else$
\setcounter{secnumdepth}{0}
$endif$
$if(verbatim-in-note)$
\VerbatimFootnotes % allows verbatim text in footnotes
$endif$
$for(header-includes)$
$header-includes$
$endfor$

$if(title)$
\title{$title$}
$endif$
$if(author)$
\author{$for(author)$$author$$sep$ \and $endfor$}
$endif$
$if(date)$
\date{$date$}
$endif$

% my page number
%\pagestyle{fancy}
%\fancyhead[RO]{Sheffield \thepage}
\fancyhead{} % empty
\fancyfoot{} % empty out the default, which is page number in center
\renewcommand{\headrulewidth}{0pt}
\pagenumbering{gobble}

\usepackage{titlesec}


% colors
\definecolor{bluehaze}{RGB}{54,95,145}
\definecolor{maroonhaze}{RGB}{148,54,52}
\definecolor{orangehaze}{RGB}{227,108,10}
\definecolor{black}{RGB}{0,0,0}
% \definecolor{lightblack}{RGB}{55,55,55}
\definecolor{bluetint}{RGB}{0,50,100}
\definecolor{bluetint2}{RGB}{0,30,60}
\definecolor{lightblack}{RGB}{75,75,75}
\definecolor{almostwhite}{RGB}{235,235,245}
%\usepackage{sectsty}% http://ctan.org/pkg/sectsty
\usepackage{xcolor}% http://ctan.org/pkg/xcolor
% \sectionfont{\color{bluehaze}\normalfont\fontsize{12}{3}\selectfont}

\usepackage{needspace}
% \titleformat{\paragraph}[leftmargin]{\normalfont\normalsize\bfseries\filleft}{\theparagraph}{15pt}{}

$if(linespread)$\linespread{$linespread$}$else$
\linespread{1.0}
$endif$

$if(titleneedspace)$
\def \titleneedspace{$titleneedspace$}
$else$
\def \titleneedspace{0cm}
$endif$
% Here I am using the needspace thing to make sure that the title headings don't
% show up on the very last line of a page.

\titleformat{\section}
  {\needspace{\titleneedspace}\normalfont\fontfamily{phv}\fontsize{12pt}{12pt}\selectfont\color{black}\bfseries}{\thesection}{1em}{\MakeUppercase}

\titleformat{\subsection}
  {\needspace{\titleneedspace}\normalfont\fontfamily{phv}\fontsize{12pt}{12pt}\selectfont\color{black}\bfseries}{\thesubsection}{1em}{}

\titleformat{\subsubsection}
  {\needspace{\titleneedspace}\normalfont\fontfamily{phv}\fontsize{12pt}{12pt}\selectfont\color{bluetint2}\bfseries}{\thesubsubsection}{1em}{}

% use runin to specify no newline
\titleformat{\paragraph}[runin]
  {\normalfont\fontfamily{phv}\fontsize{11pt}{11pt}\selectfont\color{lightblack}\itshape}{\theparagraph}{}{}


% \titlespacing{command}{left spacing}{before spacing}{after spacing}[right]
\titlespacing{\section}{0pt}{12pt plus 4pt minus 2pt}{-1pt plus 2pt minus 2pt}
\titlespacing{\subsection}{0pt}{12pt plus 40pt minus 2pt}{-2pt plus 2pt minus 2pt}
\titlespacing{\subsubsection}{0pt}{6pt plus 4pt minus 2pt}{-4pt plus 2pt minus 1pt}

% for these ones, the 'after spacing' (slot 3) corresponds to right-side spacing,
% since it's paragraph-level (no newline afterwards)
\titlespacing{\paragraph}{0pt}{1pt plus 4pt minus 2pt}{6pt plus 2pt minus 1pt}
\titlespacing{\subparagraph}{0pt}{1pt plus 4pt minus 2pt}{6pt plus 2pt minus 1pt}





%\subsectionfont{\color{maroonhaze}}
%\subsubsectionfont{\color{orangehaze}}
% end colors

% this code uses titles.sty to make a pagebreak with every header 1
% so each section will begin on a new page.
%\newcommand{\sectionbreak}{\clearpage}



\begin{document}

$if(title)$
\maketitle
$endif$

$for(include-before)$
$include-before$

$endfor$
$if(toc)$
\tableofcontents

$endif$
$body$

$if(natbib)$
$if(biblio-files)$
$if(biblio-title)$
$if(book-class)$
\renewcommand\bibname{$biblio-title$}
$else$
\renewcommand\refname{$biblio-title$}
$endif$
$endif$
\bibliography{$biblio-files$}

$endif$
$endif$
$if(biblatex)$
\printbibliography$if(biblio-title)$[title=$biblio-title$]$endif$

$endif$
$for(include-after)$
$include-after$

$endfor$



\end{document}
%\documentclass$if(fontsize)$[$fontsize$]$endif${article}
\documentclass[12pt]{article}
\usepackage{amssymb,amsmath}
\usepackage{ifxetex,ifluatex}
\ifxetex
  \usepackage{fontspec,xltxtra,xunicode}
  \defaultfontfeatures{Mapping=tex-text,Scale=MatchLowercase}
\else
  \ifluatex
    \usepackage{fontspec}
    \defaultfontfeatures{Mapping=tex-text,Scale=MatchLowercase}
  \else
    \usepackage[utf8]{inputenc}
  \fi
\fi
$if(natbib)$
\usepackage{natbib}
\bibliographystyle{plainnat}
$endif$
$if(biblatex)$
\usepackage{biblatex}
$if(biblio-files)$
\bibliography{$biblio-files$}
$endif$
$endif$
$if(lhs)$
\usepackage{listings}
\lstnewenvironment{code}{\lstset{language=Haskell,basicstyle=\small\ttfamily}}{}
$endif$
$if(verbatim-in-note)$
\usepackage{fancyvrb}
$endif$
$if(fancy-enums)$
% Redefine labelwidth for lists; otherwise, the enumerate package will cause
% markers to extend beyond the left margin.
\makeatletter\AtBeginDocument{%
  \renewcommand{\@listi}
    {\setlength{\labelwidth}{4em}}
}\makeatother
\usepackage{enumerate}
$endif$
$if(tables)$
\usepackage{ctable}
\usepackage{float} % provides the H option for float placement
$endif$
$if(url)$
\usepackage{url}
$endif$
$if(graphics)$
\usepackage{graphicx}
% We will generate all images so they have a width \maxwidth. This means
% that they will get their normal width if they fit onto the page, but
% are scaled down if they would overflow the margins.
\makeatletter
\def\maxwidth{\ifdim\Gin@nat@width>\linewidth\linewidth
\else\Gin@nat@width\fi}
\makeatother
\let\Oldincludegraphics\includegraphics
\renewcommand{\includegraphics}[1]{\Oldincludegraphics[width=\maxwidth]{#1}}
$endif$
\usepackage{caption}
\DeclareCaptionLabelFormat{nolabel}{}
\captionsetup{labelformat=nolabel}
\ifxetex
  \usepackage[setpagesize=false, % page size defined by xetex
              unicode=false, % unicode breaks when used with xetex
              xetex]{hyperref}
\else
  \usepackage[unicode=true]{hyperref}
\fi
\hypersetup{breaklinks=true, pdfborder={0 0 0}}
$if(strikeout)$
\usepackage[normalem]{ulem}
% avoid problems with \sout in headers with hyperref:
\pdfstringdefDisableCommands{\renewcommand{\sout}{}}
$endif$
$if(subscript)$
\newcommand{\textsubscr}[1]{\ensuremath{_{\scriptsize\textrm{#1}}}}
$endif$
\setlength{\parindent}{0pt}
\setlength{\parskip}{6pt plus 2pt minus 1pt}
\setlength{\emergencystretch}{3em}  % prevent overfull lines
$if(listings)$
\usepackage{listings}
$endif$
$if(numbersections)$
$else$
\setcounter{secnumdepth}{0}
$endif$
$if(verbatim-in-note)$
\VerbatimFootnotes % allows verbatim text in footnotes
$endif$
$for(header-includes)$
$header-includes$
$endfor$

$if(title)$
\title{$title$}
$endif$
$if(author)$
\author{$for(author)$$author$$sep$ \and $endfor$}
$endif$
$if(date)$
\date{$date$}
$endif$

\begin{document}
$if(title)$
\maketitle
$endif$

$for(include-before)$
$include-before$

$endfor$
$if(toc)$
\tableofcontents

$endif$
$body$

$if(natbib)$
$if(biblio-files)$
$if(biblio-title)$
$if(book-class)$
\renewcommand\bibname{$biblio-title$}
$else$
\renewcommand\refname{$biblio-title$}
$endif$
$endif$
\bibliography{$biblio-files$}

$endif$
$endif$
$if(biblatex)$
\printbibliography$if(biblio-title)$[title=$biblio-title$]$endif$

$endif$
$for(include-after)$
$include-after$

$endfor$
\end{document}
